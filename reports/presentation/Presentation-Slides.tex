%%%%%%%%%%%%%%%%%%%%%%%%%%%%%%%%%%%%%%%%%%%%
%   Based on Beamer template created by Rodrigo Platte, Arizona State University, May 18 2007.
%   Feel free to modify this file to generate your own presentation.
%
%   This Latex file should be compiled with pdflatex, for this reason, it is recommended 
%   that you convert the figures in your presentation to pdf. Postscript files (.ps and .eps) 
%   will not compile properly when pdflatex is used. You may use commands like "convert"
%   or "eps2pdf" (in linux) to convert your figures. 
%
%   The final output is a PDF file that is best visualized with Adobe Reader.
%
%   More information can be found in beameruserguide.pdf
%
%   This presentation requires the following files:
%   BEAMERoptions.tex, ASUlogo.pdf,  asu.pdf, beamerouterthememathASUlogo.sty
%%%%%%%%%%%%%%%%%%%%%%%%%%%%%%%%%%%%%%%%%%%%

\PassOptionsToPackage{table}{xcolor}
\documentclass[compress]{beamer}
% Use this instead for printing and distributing your slides (suppresses overlays)
%\documentclass[handout]{beamer}


%%%%%%%%%%%%%%%%%%%%%%%%%%%%%%%%%%%%%%%%
% Edit the file BEAMERoptions.tex to change theme, color, fonts, logo, etc.     %
\input{BEAMERoptions.tex}					                                         %
%%%%%%%%%%%%%%%%%%%%%%%%%%%%%%%%%%%%%%%%

% Load packages
%%%%%%%%%%%%%%%%%%%%%
\usepackage[english]{babel}
\usepackage{graphics}
\usepackage{colortbl}
\usepackage{multimedia} % for movies and sound
\usepackage{times}
\usepackage{multirow}
\usepackage[round]{natbib}
\usepackage[table]{xcolor}
%%%%%%%%%%%%%%%%%%%%%


%% The title and name 
%% Note: [short title]{long title}, [short author(s) name]{long author(s) name}
\title[Semi-supervised propensity score adjustment]{Semi-supervised propensity score adjustment}
\author[D. Herren]{\emph{Andrew Herren} }
\date[]{}
% Show ASU logo in title page
%%%%%%%%%%%%%%%%%%%%%%
\institute[Mathematics and Statistics]{
\includegraphics[height=.85cm]{ASUlogo.pdf} \\
{\color{ASUred} SCHOOL OF \textbf{MATHEMATICAL AND STATISTICAL SCIENCES}}}
%%%%%%%%%%%%%%%%%%%%%

%%%%%%%%%%%%%%%%%%%%%%%%%%%%%%%%%%%%%%%%%%
%%%%%%%%%%%% Presentation Starts Here %%%%%%%%%%%%%%%%
%%%%%%%%%%%%%%%%%%%%%%%%%%%%%%%%%%%%%%%%%%

\begin{document}

%%% Title frame %%%%%
\begin{frame}[plain]
	\titlepage
	\transboxout
\end{frame}

%%%% Outline (optional) %%%%%%%%%%%%%%%%%%%%%
%% The outline depends on \section and \subsection commands
%%%%%%%%%%%%%%%%%%%%%%%%%%%%%%%%%%
%\begin{frame}
%  \frametitle{Outline}
%  \tableofcontents[pausesections]
%  \transboxin
%\end{frame}
%%%%%%%%%%%%%%%%%%%%%%%%%%%%%%%%%%

\section[Problem Statement]{Problem Statement}

\begin{frame} \frametitle{Notation and problem statement}
 \begin{itemize}
  \item Let $Y$ be an outcome of interest (e.g. blood pressure)
  \item Let $Z$ be an intervention that may impact $Y$ (e.g. a new medicine)
  \item Let $X$ be a vector of covariates (e.g. demographics, health status)
  \end{itemize}

\pause

\bigskip
\bigskip

 \begin{itemize}
  \item Causal inference = concerned with counterfactuals
  \item Only observe $Y$ when $Z = 1$ or $Z = 0$
  \item Define Average Treatment Effect (ATE) as:
 \end{itemize}
 \[E_X[E[Y | X, Z = 1] - E[Y | X, Z = 0]]\]


\transboxin
\end{frame}

\begin{frame} \frametitle{Propensity scores}
Propensity score ($p(X)$) is probability of receiving treatment $(P(Z = 1 | X))$
 \begin{itemize}
  \item Assumption of \textit{\textbf{strong ignorability}}\footnote[frame]{\cite{rosenbaum1983central}}: $0 < p < 1$ and only depends on $X$
  \item $p(X)$ balances differences between treatment and control groups:
  \begin{itemize}
	\item $Y \perp Z \; | \; p(X)$
  \end{itemize}
 \end{itemize}

\pause 

\bigskip
\bigskip
\bigskip

\textbf{Importance}: many methods use $p(X)$ to estimate ATE, in two steps
  \begin{enumerate}
	\item Estimate propensity scores based on $X$
	\item Incorporate estimated propensities in ATE estimation procedure (weighting, matching, regression)
  \end{enumerate}

\transboxin
\end{frame}


\begin{frame} \frametitle{Semi-supervised learning}
Semi-supervised learning\footnote[frame]{\cite{liang2007use}}: supervised learning with 
partially observed $Y$

\medskip
\begin{center}
\begin{tabular}{ |p{1cm}|p{1cm}|p{1cm}|p{1cm}|p{1cm}|p{1cm}|p{1cm}|p{1cm}|  }
\hline
${\bf Y}$ & ${\bf Z}$ & ${\bf X_1}$ & ${\bf X_2}$ & ${\bf X_3}$ & ... & ... & ${\bf X_p}$ \\
\hline
$y_1$ & $z_1$ & $x_{11}$ & $x_{21}$ & $x_{31}$ & ... & ... & $x_{p1}$ \\
$y_2$ & $z_2$ & $x_{12}$ & $x_{22}$ & $x_{32}$ & ... & ... & $x_{p2}$ \\
... & ... & ... & ... & ... & ... & ... & ... \\
\hline
\cellcolor{lightgray} & ... & ... & ... & ... & ... & ... & ... \\
\cellcolor{lightgray} & $z_k$ & $x_{1, k}$ & $x_{2, k}$ & $x_{3, k}$ & ... & ... & $x_{p, k}$ \\
\cellcolor{lightgray} & $z_{k+1}$ & $x_{1, k+1}$ & $x_{2, k+1}$ & $x_{3, k+1}$ & ... & ... & $x_{p, k+1}$ \\
\cellcolor{lightgray} & ... & ... & ... & ... & ... & ... & ... \\
\cellcolor{lightgray} & ... & ... & ... & ... & ... & ... & ... \\
\cellcolor{lightgray} & $z_{n}$ & $x_{1, n}$ & $x_{2, n}$ & $x_{3, n}$ & ... & ... & $x_{p, n}$ \\
\hline
\end{tabular}
\end{center}

\transboxin
\end{frame}

\section[Methods]{Methods}

\begin{frame} \frametitle{How to make use of unlabeled data}
We consider three cases:
\begin{itemize}
\item Complete case analysis: labeled observations only for $p$ and ATE
\item Semi-supervised learning: estimate $p$ with unlabeled $X$ and $Z$
\item Ground truth: use the true propensities to estimate ATE
\end{itemize}

\bigskip
\bigskip
\bigskip
\bigskip
Motivation: unlabeled data will bring us closer to true propensities
\end{frame}

\begin{frame} \frametitle{ATE estimators}
\begin{itemize}
\item Inverse propensity weighting (IPW) estimator\footnote[frame]{\cite{hirano2003efficient}, \cite{cerulli2014treatrew}}
\[\widehat{\textrm{ATE}} = \frac{1}{n} \sum_{i = 1}^{n} \left( \frac{Y_i Z_i}{P(Z_i = 1 | X_i)} - \frac{Y_i (1 - Z_i)}{(1 - P(Z_i = 1 | X_i))} \right) \]
\item Targeted maximum likelihood estimator (TMLE) \footnote[frame]{\cite{van2010targeted}, \cite{van2010targeted2}, \cite{gruber2009targeted}}
\[\widehat{\textrm{ATE}} = \frac{1}{n} \sum_{i = 1}^{n} \left( Q^*\left( Z_i = 1, X_i \right) - Q^*\left( Z_i = 0, X_i \right) \right) \] 
\item Bayesian causal forests (BCF)\footnote[frame]{\cite{hahn2017bayesian}}
\[f(X_i, Z_i) = \mu(X_i, \hat{p_i}) + \tau(X_i) Z_i \]
\[\widehat{\textrm{ATE}} = \frac{1}{n} \sum_{i = 1}^{n} \left( f(X_i, Z_i = 1) - f(X_i, Z_i = 0) \right) \]
\end{itemize}
\end{frame}

\begin{frame} \frametitle{Fitting the propensity model}

When estimating propensities, we consider two approaches:

\begin{enumerate}
\item Logistic regression
\begin{itemize}
\item $\hat{p} = \frac{1}{1 + \exp(-(\hat{\mathbf{\beta}}X))}$
\item Logistic regression is only considered in the context of the IPW estimator
\end{itemize}
\item Bayesian additive regression trees (BART) (\cite{chipman2010bart})
\begin{itemize}
\item $\hat{p}$ estimated by a nonparametric Bayesian tree ensemble
\end{itemize}
\end{enumerate}

\end{frame}

\begin{frame} \frametitle{Data generating processes}
We simulate data according to the following relationship between $X$, $Y$, and $Z$:

\begin{align*}
Y &= \mu(X) + \tau Z + \varepsilon\\
\mu(X) &= 
\begin{cases}
3 + x_1 x_3,&  x_5 = 1,\\
x_1 x_3,&  x_5 = 2\\
-3 + x_1 x_3,&  x_5 = 3
\end{cases} \\
x_1, x_2, x_3 &\sim \mathcal{N}(0, 1)\\
x_4 &\sim \textrm{Bernoulli}(0.5)\\
x_5 &\sim \textrm{Categorical}(0.25, 0.5, 0.25)\\
\tau &= 3\\
\varepsilon &\sim \mathcal{N}(0, 1)\\
\end{align*}

\transboxin
\end{frame}

\begin{frame} \frametitle{Propensity model}
We consider two cases for obtaining simulated propensities:

\begin{enumerate}
\item Randomized treatment assignment: $P(Z = 1 | X) = 0.5$
\item Targeted selection\footnote[frame]{\cite{hahn2017bayesian}}: $P(Z = 1 | X) = 0.8\Phi(3 \mu(X) / s - 0.5 x_1) + 0.05 + u / 10$
\begin{itemize}
	\item $\Phi(\cdot)$ is the standard normal CDF
	\item $s$ is the sample standard deviation of simulated observation of $\mu(X)$
	\item $u \sim \mathrm{Uniform}(0, 1)$ 
\end{itemize}
\end{enumerate}

\bigskip
Targeted selection is challenging for most causal inference methods
\end{frame}

\section[Simulation results]{Simulation results}

\begin{frame} \frametitle{Simulation evaluations}
We evaluate the following metrics

\begin{itemize}
	\item \textbf{RMSE}: root mean squared error of estimated treatment effects
	\item \textbf{Bias}: difference between true ATE (3) and average estimated ATE
	\item \textbf{Coverage}: share of 95\% confidence intervals that contain the true ATE
\end{itemize}

\transboxin
\end{frame}

\begin{frame}
\vfill
\centering
\begin{beamercolorbox}[sep=8pt,center,shadow=true,rounded=true]{title}
\usebeamerfont{title}Targeted selection\par%
\end{beamercolorbox}
\vfill
\end{frame}

\begin{frame} \frametitle{Simulation evaluations}
These simulations are drawn assuming targeted selection

\begin{itemize}
	\item 90\% of the data missing outcome labels
	\item Outcomes are unlabeled completely at random (MCAR)
	\item Sample sizes range from 500, 1,000, 5,000
	\item Treatment effects homogeneous ($\tau=3$)
\end{itemize}

\transboxin
\end{frame}


\begin{frame} \frametitle{How to read these tables}
Focus attention on bias and coverage metrics
\begin{itemize}
	\item Bias should be close to 0
	\item Interval coverage should be close to 95\%
\end{itemize}

Table rows: increasing order of unlabeled data usage / ground truth

\begin{table}[ht]
\centering
\begingroup\small
\begin{tabular}{lllcc>{\columncolor[gray]{.9}}c>{\columncolor[gray]{.9}}c}
  \hline
Approach & N & \# labeled & ATE & RMSE & Bias & Coverage \\ 
  \hline
Complete Case &  500 & 50 & $\cdot$ & $\cdot$ & $\cdot$ & $\cdot$ \\ 
  Complete Case & 1,000 & 100 & $\cdot$ & $\cdot$ & $\cdot$ & $\cdot$ \\ 
  Complete Case & 5,000 & 500 & $\cdot$ & $\cdot$ & $\cdot$ & $\cdot$ \\ 
   \rowcolor[gray]{.8} Semi-supervised &   500 &  50 & $\cdot$ & $\cdot$ & $\cdot$ & $\cdot$ \\ 
   \rowcolor[gray]{.7} Semi-supervised & 1,000 & 100 & $\cdot$ & $\cdot$ & $\cdot$ & $\cdot$ \\ 
   \rowcolor[gray]{.6} Semi-supervised & 5,000 & 500 & $\cdot$ & $\cdot$ & $\cdot$ & $\cdot$ \\ 
   \rowcolor[gray]{.5} True propensities & 5,000 & 500 & $\cdot$ & $\cdot$ & $\cdot$ & $\cdot$ \\ 
   \hline
\end{tabular}
\endgroup
\caption{n simulations with outcomes MCAR} 
\end{table}

\transboxin
\end{frame}

\begin{frame} \frametitle{IPW / logistic regression with targeted selection}
Data: Homogeneous treatment effects with targeted selection

Model: IPW estimator, with propensities fit using logistic regression

% latex table generated in R 3.6.2 by xtable 1.8-4 package
% Sat Apr  4 06:53:31 2020
\begin{table}[ht]
\centering
\begingroup\small
\begin{tabular}{lllcc>{\columncolor[gray]{.9}}c>{\columncolor[gray]{.9}}c}
  \hline
Approach & N & \# labeled & ATE & RMSE & Bias & Coverage \\ 
  \hline
  Complete Case & 5,000 & 500 & 3.50 & 0.51 & 0.50 & 0.59 \\ 
  Complete Case & 5,000 & 100 & 3.48 & 0.68 & 0.48 & 0.77 \\ 
  Complete Case & 5,000 &  50 & 3.49 & 0.89 & 0.49 & 0.67 \\ 
   \rowcolor[gray]{.8} Semi-supervised & 5,000 &  50 & 3.48 & 0.97 & 0.48 & 0.67 \\ 
   \rowcolor[gray]{.7} Semi-supervised & 5,000 & 100 & 3.56 & 0.73 & 0.56 & 0.72 \\ 
   \rowcolor[gray]{.6} Semi-supervised & 5,000 & 500 & 3.50 & 0.52 & 0.50 & 0.55 \\ 
   \rowcolor[gray]{.5} True propensities & 5,000 & 500 & 2.99 & 0.25 & -0.01 & 0.89 \\ 
   \hline
\end{tabular}
\endgroup
\caption{500 simulations with outcomes MCAR} 
\end{table}


A mis-specified propensity model results in \textbf{badly} biased ATE estimates
and poor coverage, and unlabeled data not helpful

%\input{../../outputs/simulations_IPW_LIN_HOM_MAR_Linear_90.tex}
\transboxin
\end{frame}

\begin{frame} \frametitle{IPW / BART with targeted selection}
Data: Homogeneous treatment effects with targeted selection

Model: IPW estimator, with propensities fit using BART

% latex table generated in R 3.6.2 by xtable 1.8-4 package
% Sat Apr  4 06:53:31 2020
\begin{table}[ht]
\centering
\begingroup\small
\begin{tabular}{lllcc>{\columncolor[gray]{.9}}c>{\columncolor[gray]{.9}}c}
  \hline
Approach & N & \# labeled & ATE & RMSE & Bias & Coverage \\ 
  \hline
  Complete Case & 5,000 &  50 & 3.90 & 0.90 & 0.90 & 0.47 \\ 
  Complete Case & 5,000 & 100 & 3.79 & 0.79 & 0.79 & 0.39 \\ 
Complete Case & 5,000 & 500 & 3.50 & 0.50 & 0.50 & 0.36 \\ 
   \rowcolor[gray]{.8} Semi-supervised & 5,000 &  50 & 3.00 & 0.85 & -0.00 & 0.73 \\ 
   \rowcolor[gray]{.7} Semi-supervised & 5,000 & 100 & 3.00 & 0.58 & -0.00 & 0.86 \\ 
   \rowcolor[gray]{.6} Semi-supervised & 5,000 & 500 & 3.00 & 0.26 & -0.00 & 0.90 \\ 
   \rowcolor[gray]{.5} True propensities & 5,000 & 500 & 2.99 & 0.25 & -0.01 & 0.89 \\ 
   \hline
\end{tabular}
\endgroup
\caption{500 simulations with outcomes MCAR} 
\end{table}


Properly specified propensity scores improve bias and interval coverage

\transboxin
\end{frame}

\begin{frame} \frametitle{TMLE / BART with targeted selection}
Data: Homogeneous treatment effects with targeted selection

Model: TMLE, with BART for propensity and outcome model

% latex table generated in R 3.6.2 by xtable 1.8-4 package
% Sat Apr  4 06:53:31 2020
\begin{table}[ht]
\centering
\begingroup\small
\begin{tabular}{lllcc>{\columncolor[gray]{.9}}c>{\columncolor[gray]{.9}}c}
  \hline
Approach & N & \# labeled & ATE & RMSE & Bias & Coverage \\ 
  \hline
  Complete Case & 5,000 &  50 & 3.58 & 0.62 & 0.58 & 0.35 \\ 
  Complete Case & 5,000 & 100 & 3.44 & 0.47 & 0.44 & 0.33 \\ 
  Complete Case & 5,000 & 500 & 3.12 & 0.15 & 0.12 & 0.63 \\ 
   \rowcolor[gray]{.8} Semi-supervised & 5,000 &  50 & 3.33 & 0.51 & 0.33 & 0.60 \\ 
   \rowcolor[gray]{.7} Semi-supervised & 5,000 & 100 & 3.24 & 0.35 & 0.24 & 0.64 \\ 
   \rowcolor[gray]{.6} Semi-supervised & 5,000 & 500 & 3.07 & 0.13 & 0.07 & 0.80 \\ 
   \rowcolor[gray]{.5} True propensities & 5,000 & 500 & 3.05 & 0.12 & 0.05 & 0.84 \\ 
   \hline
\end{tabular}
\endgroup
\caption{500 simulations with outcomes MCAR} 
\end{table}


TMLE provides roughly unbiased results with larger N, but poor interval coverage

\transboxin
\end{frame}

\begin{frame} \frametitle{BCF with targeted selection}
Data: Homogeneous treatment effects with targeted selection

Model: Bayesian causal forests (BCF)

% latex table generated in R 3.6.2 by xtable 1.8-4 package
% Sat Apr  4 06:53:31 2020
\begin{table}[ht]
\centering
\begingroup\small
\begin{tabular}{lllcc>{\columncolor[gray]{.9}}c>{\columncolor[gray]{.9}}c}
  \hline
Approach & N & \# labeled & ATE & RMSE & Bias & Coverage \\ 
  \hline
  Complete Case & 5,000 &  50 & 3.54 & 0.99 & 0.54 & 0.70 \\ 
  Complete Case & 5,000 & 100 & 3.43 & 0.80 & 0.43 & 0.80 \\ 
  Complete Case & 5,000 & 500 & 3.06 & 0.37 & 0.06 & 0.96 \\ 
  \rowcolor[gray]{.8} Semi-supervised & 5,000 &  50 & 3.26 & 0.80 & 0.26 & 0.77 \\ 
  \rowcolor[gray]{.7} Semi-supervised & 5,000 & 100 & 3.15 & 0.58 & 0.15 & 0.86 \\ 
  \rowcolor[gray]{.6} Semi-supervised & 5,000 & 500 & 3.03 & 0.37 & 0.03 & 0.98 \\ 
  \rowcolor[gray]{.5} True propensities & 5,000 & 500 & 3.01 & 0.32 & 0.01 & 0.98 \\ 
  \hline
\end{tabular}
\endgroup
\caption{500 simulations with outcomes MCAR} 
\end{table}


BCF works best and provides strong coverage

\transboxin
\end{frame}

\begin{frame}
\vfill
\centering
\begin{beamercolorbox}[sep=8pt,center,shadow=true,rounded=true]{title}
\usebeamerfont{title}Randomized treatment assignment\par
\end{beamercolorbox}
\vfill
\end{frame}

\begin{frame} \frametitle{Simulation evaluations}
These simulations are drawn assuming randomized treatment assignment

\begin{itemize}
	\item 90\% of the data missing outcome labels
	\item Outcomes are unlabeled completely at random (MCAR)
	\item Sample sizes range from 500, 1,000, 5,000
	\item Treatment effects homogeneous ($\tau=3$)
\end{itemize}

\transboxin
\end{frame}

\begin{frame} \frametitle{Complete case estimates with randomized treatment assignment}
In the randomized assignment case, all methods work quite well

\input{../../outputs/simulations_CC_ALL_EXP_LIN_HOM_MCAR_90_HIGHLIGHT.tex}

\transboxin
\end{frame}

\begin{frame} \frametitle{Semi-supervised estimates with randomized treatment assignment}
But unlabeled data still helpful for IPW estimates

\input{../../outputs/simulations_SSL_ALL_EXP_LIN_HOM_MCAR_90_HIGHLIGHT.tex}

\transboxin
\end{frame}

\begin{frame} \frametitle{Conclusion}
 \begin{itemize}
  \item No amount of unlabeled data will fix a mis-specified propensity model
  \begin{itemize}
  \item Can even harm inferences by reducing variance around a biased estimate
  \end{itemize}
  \item BCF and IPW/BART both work best on targeted selection data
  \item Learning ATE with experimental data much easier
 \end{itemize}
\transboxin
\end{frame}

\begin{frame} \frametitle{Future work}
 \begin{itemize}
  \item Articulate results using semi-supervised learning theory\footnote[frame]{\cite{belkin2006manifold}}
  \item Complex outcome functions
  \item Heterogeneous treatment effects
  \item Unlabeled outcomes correlated with covariates 
  \item Connection to survey inference / superpopulation modeling
 \end{itemize}
\transboxin
\end{frame}

\begin{frame} \frametitle{}
 \begin{center}
	\Huge Questions?
 \end{center}
\transboxin
\end{frame}

\begin{frame}[allowframebreaks] \frametitle{References}
\bibliographystyle{plainnat} 
\bibliography{Presentation-Slides}
\end{frame}

\end{document}
